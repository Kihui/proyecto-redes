\documentclass[12pt]{article}
\usepackage[left=2cm,right=2cm,top=2cm,bottom=2cm,letterpaper]{geometry}
\usepackage{lmodern}
\usepackage[T1]{fontenc}
\usepackage[utf8]{inputenc}
\usepackage[spanish,activeacute]{babel}
\usepackage{hyperref}
\usepackage{graphicx}
\graphicspath{{media/}}
\usepackage{float}
\usepackage{caption}
\usepackage[toc]{multitoc}
\setcounter{tocdepth}{2}
% automata
\usepackage{tikz}
\title{Proyecto 2}
\author{Carlos Gerardo Acosta Hernández \\ Andrea Itzel González Vargas \\ Luis Pablo Mayo Vega}
\date{Redes de Computadoras\\Facultad de Ciencias, UNAM}
%\setlength{\parindent}{0em}
\begin{document}
\maketitle
\tableofcontents
\newpage
\section{Especificación de requerimientos}
\subsection{Enunciado del problema}
\subsection{Objetivo de la aplicación}
\subsubsection{Casos de uso}
\section{Diseño del protocolo}
\subsection{Máquina de Estados Finita}
Máquina de Estados Finita para el protocolo de la capa de aplicación.
\begin{center}
\begin{tikzpicture}[scale=0.2]
  \tikzstyle{every node}+=[inner sep=0pt]
  \draw [black] (16.9,-20.4) circle (3);
  \draw (16.9,-20.4) node {$q_1$};
  \draw [black] (25.6,-11.5) circle (3);
  \draw (25.6,-11.5) node {$q_2$};
  \draw [black] (40.8,-6.2) circle (3);
  \draw (40.8,-6.2) node {$q_3$};
  \draw [black] (45.4,-20.4) circle (3);
  \draw (45.4,-20.4) node {$q_4$};
  \draw [black] (58.2,-15.1) circle (3);
  \draw (58.2,-15.1) node {$q_5$};
  \draw [black] (47.3,-32.8) circle (3);
  \draw (47.3,-32.8) node {$q_6$};
  \draw [black] (60.8,-30.3) circle (3);
  \draw (60.8,-30.3) node {$q_7$};
  \draw [black] (60.8,-45.4) circle (3);
  \draw (60.8,-45.4) node {$q_8$};
  \draw [black] (29.8,-32.8) circle (3);
  \draw (29.8,-32.8) node {$q_9$};
  \draw [black] (6.8,-30.3) circle (3);
  \draw (6.8,-30.3) node {$q_0$};
  \draw [black] (19,-18.25) -- (23.5,-13.65);
  \fill [black] (23.5,-13.65) -- (22.59,-13.87) -- (23.3,-14.57);
  \draw (20.72,-14.48) node [left] {$c:02$};
  \draw [black] (41.72,-9.05) -- (44.48,-17.55);
  \fill [black] (44.48,-17.55) -- (44.7,-16.63) -- (43.75,-16.94);
  \draw (43.87,-12.62) node [right] {$c:20$};
  \draw [black] (48.17,-19.25) -- (55.43,-16.25);
  \fill [black] (55.43,-16.25) -- (54.5,-16.09) -- (54.88,-17.02);
  \draw (54.37,-18.29) node [below] {$c:12$};
  \draw [black] (45.85,-23.37) -- (46.85,-29.83);
  \fill [black] (46.85,-29.83) -- (47.22,-28.97) -- (46.23,-29.12);
  \draw (45.65,-26.79) node [left] {$c:13$};
  \draw [black] (50.25,-32.25) -- (57.85,-30.85);
  \fill [black] (57.85,-30.85) -- (56.97,-30.5) -- (57.15,-31.48);
  \draw (52.72,-30.81) node [above] {$s:22$};
  \draw [black] (62.285,-32.899) arc (22.9704:-22.9704:12.687);
  \fill [black] (62.28,-32.9) -- (62.14,-33.83) -- (63.06,-33.44);
  \draw (63.79,-37.85) node [right] {$s:23$};
  \draw [black] (58.977,-43.03) arc (-150.57259:-209.42741:10.544);
  \fill [black] (58.98,-43.03) -- (59.02,-42.09) -- (58.15,-42.58);
  \draw (57.12,-37.85) node [left] {$c:15$};
  \draw [black] (39.002,-8.592) arc (-43.919:-97.63528:12.314);
  \fill [black] (39,-8.59) -- (38.09,-8.82) -- (38.81,-9.51);
  \draw (36.62,-12.26) node [below] {$c:10$};
  \draw [black] (27.177,-8.959) arc (140.39464:78.05107:11.056);
  \fill [black] (27.18,-8.96) -- (28.07,-8.66) -- (27.3,-8.02);
  \draw (29.63,-4.98) node [above] {$c:11$};
  \draw [black] (26.18,-14.44) -- (29.22,-29.86);
  \fill [black] (29.22,-29.86) -- (29.56,-28.98) -- (28.57,-29.17);
  \draw (28.44,-21.84) node [right] {$c:00$};
  \draw [black] (8.94,-28.2) -- (14.76,-22.5);
  \fill [black] (14.76,-22.5) -- (13.84,-22.7) -- (14.54,-23.42);
  \draw (14.54,-25.83) node [below] {$c:01$};
\end{tikzpicture}
\\

\subsection{Descripción de los estados}
\begin{tabular}{|l|p{9cm}|}
  \hline
  Estado & Descripción \\
  \hline
  $q_0$ & Estado inicial para la conexión de la aplicación. \\ \hline
  $q_1$ & Estado de \textit{Acknowledgment} del servidor con el cliente. \\ \hline
  $q_2$ & Estado donde la conexión se ha realizado con éxito. \\ \hline
  $q_3$ & Estado de inicio de sesión para un cliente. \\ \hline
  $q_4$ & Estado de inicio de sesión exitoso, el servidor está en espera de la siguiente acción de parte del cliente. También se muestra la imagen del pokémon recien capturado después de una captura exitosa. \\ \hline
  $q_5$ & Estado donde el usuario con la sesión activa decide revisar su \textit{Pokédex}. \\ \hline
  $q_6$ & Estado donde el usuario con la sesión activa decide capturar un pokémon.\\ \hline
  $q_7$ & Estado donde el servidor recibe la solicitud del cliente y le ofrece un pokémon aleatoriamente.\\ \hline
  $q_8$ & Estado donde el usuario acepta o no capturar el pokemon ofrecido por el servidor.\\ \hline
  $q_9$ & Cierre de conexión. \\
  \hline
\end{tabular}
\end{center}
\newpage

\subsection{Descripción de los mensajes en la comunicación cliente-servidor}
\begin{center}
  \begin{tabular}{|l|c|p{5.9cm}|}
    \hline
    Código & Segmento & Descripción \\ \hline
    \hline
    00 & \texttt{|code|num\_con|} & Termina la conexión identificada con num\_con. \\ \hline
    01 & \texttt{|code|} & SYN del cliente (parte del two-way handshake del protocolo para iniciar la comunicación entre el cliente y el servidor). \\ \hline
    02 & \texttt{|code|num\_con|} & SYN/ACK del servidor (parte del two-way handshake) \\ \hline
    10 & \texttt{|code|num\_con|name|} & Solicitud de inicio de sesión del cliente. El parámetro \texttt{name} se refiere al nombre de usuario de la persona conectándose a través del cliente y su tamaño está en el rango de 1 a 32 bytes. \\ \hline
    11 & \texttt{|code|num\_con|} & Solicitud de cierre de sesión del cliente. \\ \hline
    12 & \texttt{|code|num\_con|name|} & Consulta del usuario a su \textit{Pokédex}. \\ \hline
    13 & \texttt{|code|num\_con|} & Menú de captura. \\ \hline
    14 & \texttt{|code|num\_con|} & El usuario rechaza el pokémon ofrecido aleatoriamente. \\ \hline
    15 & \texttt{|code|num\_con|} & Intento de captura del pokémon ofrecido aleatoriamente. \\ \hline
    20 & \texttt{|code|} & inicio de sesión exitoso. \\ \hline
    21 & \texttt{|code|long|nombre|longitud|imagen} & Pokédex query. \\ \hline %wat
    22 & \texttt{|code|name|num\_intentos\_restantes} & funcionamiento aleatorio. \\ \hline
    23 & \texttt{|code|} & Pokémon no capturado. \\ \hline
    24 & \texttt{|code| "2i"|} & Pokémon capturado. \\ \hline
    40 & \texttt{|code|} & Invalid Name Error. \\ \hline
    41 & \texttt{|code|} & Maximum Number of Attempts Reached. \\ \hline
    44 & \texttt{|code|} & Not Found Error. \\ \hline
    45 & \texttt{|code|} & Malformación. \\ \hline
    41 & \texttt{|code|} & No existe conexión. \\ \hline
    47 & \texttt{|code|} & Sin sesión. \\
    \hline
  \end{tabular}
\end{center}

\subsection{Diseño de la base de datos}

\section{Implementación del protocolo}
\subsection{Especificación del ambiente de desarrollo}
\subsection{Diagrama de clases (Maybe no, pero al menos mención de la estructura programática del proyecto, lo que sea más fácil jijijijijijij)}

\section{Uso y pruebas del protocolo}
\subsection{Manual de uso}
\subsection{Demostración del funcionamiento (por caso de uso ijijisjiajij)}


\end{document}
