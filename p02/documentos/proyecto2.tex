\documentclass[12pt]{article}
\usepackage[left=2cm,right=2cm,top=2cm,bottom=2cm,letterpaper]{geometry}
\usepackage{lmodern}
\usepackage[T1]{fontenc}
\usepackage[utf8]{inputenc}
\usepackage[spanish,activeacute]{babel}
\usepackage{hyperref}
\usepackage{graphicx}
\graphicspath{{media/}}
\usepackage{float}
\usepackage{caption}
\usepackage[toc]{multitoc}
\setcounter{tocdepth}{2}
% automata
\usepackage{tikz}
\title{Proyecto 2}
\author{Carlos Gerardo Acosta Hernández \\ Andrea Itzel González Vargas \\ Luis Pablo Mayo Vega}
\date{Redes de Computadoras\\Facultad de Ciencias, UNAM}
%\setlength{\parindent}{0em}
\begin{document}
\maketitle
\tableofcontents
\newpage
\section{Especificación de requerimientos}
\subsection{Enunciado del problema}
\subsection{Objetivo de la aplicación}
\subsubsection{Casos de uso}
\section{Diseño del protocolo}
\subsection{Máquina de Estados Finita}
Máquina de Estados Finita para el protocolo de la capa de aplicación.
\begin{center}
\begin{tikzpicture}[scale=0.2]
\tikzstyle{every node}+=[inner sep=0pt]
\draw [black] (7.7,-22) circle (3);
\draw (7.7,-22) node {$q_0$};
\draw [black] (21.9,-21.4) circle (3);
\draw (21.9,-21.4) node {$q_1$};
\draw [black] (35.4,-21.4) circle (3);
\draw (35.4,-21.4) node {$q_2$};
\draw [black] (63.3,-21.4) circle (3);
\draw (63.3,-21.4) node {$q_3$};
\draw [black] (35.4,-7.3) circle (3);
\draw (35.4,-7.3) node {$q_4$};
\draw [black] (63.3,-46.6) circle (3);
\draw (63.3,-46.6) node {$q_5$};
\draw [black] (35.4,-46.6) circle (3);
\draw (35.4,-46.6) node {$q_6$};
\draw [black] (7.7,-49.9) circle (3);
\draw (7.7,-49.9) node {$q_9$};
\draw [black] (9.968,-20.054) arc (121.77715:63.06187:9.706);
\fill [black] (19.48,-19.65) -- (18.99,-18.84) -- (18.54,-19.74);
\draw (14.59,-18.02) node [above] {$c:10$};
\draw [black] (24.9,-21.4) -- (32.4,-21.4);
\fill [black] (32.4,-21.4) -- (31.6,-20.9) -- (31.6,-21.9);
\draw (28.65,-21.9) node [below] {$s:20$};
\draw [black] (19.52,-23.209) arc (-61.01633:-114.14465:10.403);
\fill [black] (10.22,-23.6) -- (10.75,-24.39) -- (11.16,-23.47);
\draw (15,-25.09) node [below] {$s:44$};
\draw [black] (34.348,-24.204) arc (-26.53097:-150.98729:14.4);
\fill [black] (8.87,-24.76) -- (8.82,-25.7) -- (9.7,-25.21);
\draw (21.8,-32.71) node [below] {$c:11$};
\draw [black] (37.033,-9.805) arc (25.08093:-25.08093:10.722);
\fill [black] (37.03,-9.81) -- (36.92,-10.74) -- (37.82,-10.32);
\draw (38.54,-14.35) node [right] {$c:12$};
\draw [black] (34.029,-18.74) arc (-159.56755:-200.43245:12.574);
\fill [black] (34.03,-18.74) -- (34.22,-17.82) -- (33.28,-18.16);
\draw (32.74,-14.35) node [left] {$s:21/e:44$};
\draw [black] (38.4,-21.4) -- (60.3,-21.4);
\fill [black] (60.3,-21.4) -- (59.5,-20.9) -- (59.5,-21.9);
\draw (49.35,-21.9) node [below] {$c:13$};
\draw [black] (63.3,-24.4) -- (63.3,-43.6);
\fill [black] (63.3,-43.6) -- (63.8,-42.8) -- (62.8,-42.8);
\draw (62.8,-34) node [left] {$s:22$};
\draw [black] (38.199,-45.523) arc (108.58634:71.41366:34.985);
\fill [black] (38.2,-45.52) -- (39.12,-45.74) -- (38.8,-44.79);
\draw (49.35,-43.2) node [above] {$c:15$};
\draw [black] (60.488,-47.642) arc (-72.05358:-107.94642:36.146);
\fill [black] (60.49,-47.64) -- (59.57,-47.41) -- (59.88,-48.36);
\draw (49.35,-49.9) node [below] {$s:23$};
\draw [black] (35.4,-43.6) -- (35.4,-24.4);
\fill [black] (35.4,-24.4) -- (34.9,-25.2) -- (35.9,-25.2);
\draw (35.9,-34) node [right] {$s:24/e:41$};
\draw [black] (7.7,-25) -- (7.7,-46.9);
\fill [black] (7.7,-46.9) -- (8.2,-46.1) -- (7.2,-46.1);
\draw (7.2,-35.95) node [left] {$c:00$};
\draw [black] (61.07,-44.59) -- (37.63,-23.41);
\fill [black] (37.63,-23.41) -- (37.88,-24.32) -- (38.56,-23.58);
\draw (51.59,-33.51) node [above] {$c:14$};
\end{tikzpicture}
\end{center}


\subsection{Descripción de los estados}
\begin{center}
\begin{tabular}{|l|p{9cm}|}
  \hline
  Estado & Descripción \\
  \hline
  $q_0$ & Conexión establecida, inicio de aplicación. \\ \hline
  $q_1$ & Inicio de sesión. \\ \hline
  $q_2$ & Menú de juego. \\ \hline
  $q_3$ & Solicitud de captura de \textit{pókemon}. \\ \hline
  $q_4$ & Búsqueda de un pókemon en la \textit{pókedex}. \\ \hline
  $q_5$ & Aparición de un \textit{Pókemon} salvaje. \\ \hline
  $q_6$ & Intento de captura de \textit{pókemon}.\\ \hline
  $q_9$ & Cierre de conexión. \\
  \hline
\end{tabular}
\end{center}
\newpage

\subsection{Descripción de los mensajes en la comunicación cliente-servidor}
\begin{center}
  \begin{tabular}{|l|c|p{5.9cm}|}
    \hline
    Código & Segmento & Descripción \\ \hline
    \hline
    00 & \texttt{|code|num\_con|} & Termina la conexión identificada con num\_con. \\ \hline
    10 & \texttt{|code|num\_con|name|} & Solicitud de inicio de sesión del cliente. El parámetro \texttt{name} se refiere al nombre de usuario de la persona conectándose a través del cliente y su tamaño está en el rango de 1 a 32 bytes. \\ \hline
    11 & \texttt{|code|num\_con|} & Solicitud de cierre de sesión del cliente. \\ \hline
    12 & \texttt{|code|num\_con|name|} & Consulta del usuario a su \textit{Pokédex}. \\ \hline
    13 & \texttt{|code|num\_con|} & Menú de captura. \\ \hline
    14 & \texttt{|code|num\_con|} & El usuario rechaza el pokémon ofrecido aleatoriamente. \\ \hline
    15 & \texttt{|code|num\_con|} & Intento de captura del pokémon ofrecido aleatoriamente. \\ \hline
    20 & \texttt{|code|} & inicio de sesión exitoso. \\ \hline
    21 & \texttt{|code|long|nombre|longitud|imagen} & Pokédex query. \\ \hline %wat
    22 & \texttt{|code|name|num\_intentos\_restantes} & funcionamiento aleatorio. \\ \hline
    23 & \texttt{|code|} & Pokémon no capturado. \\ \hline
    24 & \texttt{|code| "2i"|} & Pokémon capturado. \\ \hline
    41 & \texttt{|code|} & Maximum Number of Attempts Reached. \\ \hline
    44 & \texttt{|code|} & Not Found Error. \\ \hline
    46 & \texttt{|code|} & No existe conexión. \\ \hline
    47 & \texttt{|code|} & Sin sesión. \\
    \hline
  \end{tabular}
\end{center}

\subsection{Diseño de la base de datos}

\section{Implementación del protocolo}
\subsection{Especificación del ambiente de desarrollo}
\subsection{Diagrama de clases (Maybe no, pero al menos mención de la estructura programática del proyecto, lo que sea más fácil jijijijijijij)}

\section{Uso y pruebas del protocolo}
\subsection{Manual de uso}
\subsection{Demostración del funcionamiento (por caso de uso ijijisjiajij)}


\end{document}
